%%%%%%%%%%%%%%%%%%%%%%%%%%%%%%%%%%%%%%%%%%%%%%%%%%%%%%%%%%%%%%%%%%%%%%%%%%%%%%%%
%2345678901234567890123456789012345678901234567890123456789012345678901234567890
%        1         2         3         4         5         6         7         8

\documentclass[letterpaper, 10 pt, conference]{ieeeconf}  % Comment this line out if you need a4paper

%\documentclass[a4paper, 10pt, conference]{ieeeconf}      % Use this line for a4 paper

\IEEEoverridecommandlockouts                              % This command is only needed if 
                                                          % you want to use the \thanks command

\overrideIEEEmargins                                      % Needed to meet printer requirements.

%In case you encounter the following error:
%Error 1010 The PDF file may be corrupt (unable to open PDF file) OR
%Error 1000 An error occurred while parsing a contents stream. Unable to analyze the PDF file.
%This is a known problem with pdfLaTeX conversion filter. The file cannot be opened with acrobat reader
%Please use one of the alternatives below to circumvent this error by uncommenting one or the other
%\pdfobjcompresslevel=0
%\pdfminorversion=4

% See the \addtolength command later in the file to balance the column lengths
% on the last page of the document

% The following packages can be found on http:\\www.ctan.org
\usepackage{graphics} % for pdf, bitmapped graphics files
\usepackage{epsfig} % for postscript grcaphics files
%\usepackage{mathptmx} % assumes new font selection scheme installed
%\usepackage{times} % assumes new font selection scheme installed
\usepackage{amsmath} % assumes amsmath package installed
\usepackage{amssymb}  % assumes amsmath package installed

% パッケージ達
\usepackage{color}
% \usepackage[dvipdfmx]{graphicx} % [dvipdfmx]があると画像が表示されない.軽く調べたところdvipdfmxは日本独自のガラパゴスなのであまり使用しない方が良いかも...
% \usepackage[dvipdfmx]{color}
\usepackage{bm}
\usepackage{fancyhdr}
%\usepackage{nidanfloat}
\usepackage{float}
\usepackage{booktabs}
\usepackage{balance}
\usepackage{flushend}
\usepackage{subfigure}
\usepackage{mathtools}
\usepackage[maxfloats=256]{morefloats}
\usepackage{caption,setspace}
\usepackage{cite}
\usepackage{ikuo}
\newcommand{\FIGDIR}{./fig}	%図を置くディレクトリを指定する


\title{\LARGE \bf
        Presenting the sensation of flying with flapping virtual wings independent of the limbs
}

\author{Ken Endo$^{1}$ and Ikuo Mizuuchi$^{1}$% <-this % stops a space

        %%%%%%%%%% これいる? %%%%%%%%%%%%
        % \thanks{*This work was not supported by any organization}% <-this % stops a space
        \thanks{$^{1}$ Department of Mechanical Systems Engineering, Tokyo University of Agriculture and Technology, 2-24-16, Naka-cho, Koganei-city, Tokyo, Japan
                {\tt\small \{ken,ikuo\}@mizuuchi.lab.tuat.ac.jp}}%
        % \thanks{$^{2}$ Graduate School of Engineering, Tokyo University of Agriculture and Technology, Japan
                % {\tt\small mizuuchi@cc.tuat.ac.jp}}%
}


\begin{document}

\maketitle
\thispagestyle{empty}
\pagestyle{empty}


%%%%%%%%%%%%%%%%%%%%%%%%%%%%%%%%%%%%%%%%%%%%%%%%%%%%%%%%%%%%%%%%%%%%%%%%%%%%%%%%
\begin{abstract}
        Since ancient times, people have longed to fly in the sky.  
        Actual flying involves risks and costs, but using a VR device makes it easy to experience flight.  
        % In this research, we propose a method of presenting the sensation of flying with flapping virtual wings independent of the limbs, such as a flying lizard.  
        In this research, we proposed a method of presenting the sensation of manipulating the wings without using the limbs and a method of transmitting the force acting on the wing to humans.  
        Unlike studies that presents the sensation of flapping wings by moving the arms, new applications that use the limbs during the flight experience can be expected by flying without moving the limbs.  
        % We conducted experiments using these methods and obtained subjective evaluations.  
        % 
        From the experiment, it was confirmed that the operation by static muscle contraction is also effective for operationing wings.  
        It was also shown that the haptics presentation using EMS has a higher overall evaluation.  
        Finally, we obtained the result that the body image expansion of the virtual wings which proposed in this research is possible.
\end{abstract}


%%%%%%%%%%%%%%%%%%%%%%%%%%%%%%%%%%%%%%%%%%%%%%%%%%%%%%%%%%%%%%%%%%%%%%%%%%%%%%%%
\section{INTRODUCTION}

        % \fig{WingMan.pdf}{width=.9\hsize}{Flying with flapping virtual wings independent of the limbs}

        Since ancient times, people have longed to fly in the sky.  
        Until today, we have had a flight experience by using vehicles such as airplanes and hang gliders.  
        However, actual flying involves risks such as crashes and costs such as fuel.
        By using VR system, these can be avoided, and makes it easy to experience flight.  

        Many studies have been conducted to give a sensation of "flying" using VR devices.  
        Research on the sensation of falling generated by visual stimuli
        % \cite{奥川夏輝2017VR空間における視覚刺激によって発生する落下感覚の分析}
        and a proposal for a flight experience device using a body assistance mechanism
        % \cite{鈴木拓馬2014hmd}
        are examples.  

        Regarding research that gives a sensation of "flying with flapping", research has been conducted on a device that allows the user to board a large control device and experience a bird in flight\cite{rheiner2014birdly}.
        This method has disadvantages such as the need for a large scale device and the limitation of limbs movement.  
        In addition, there are still few studies on giving the sensation of flying with flapping one's wings.  
        In general, studies on giving the sensation of flying by becoming a bird have been conducted, and studies on giving the sensation of flying by becoming a creature with wings independent of its limbs, such as a flying lizard, have not yet been focused on.


        Fig. 1 shows how they flying  with flapping virtual wings independent of the limbs.  
        In this research, we propose a method of presenting the sensation of flying with flapping virtual wings independent of the limbs 
        % as a creature with wings growing from the back of a human, 
        as shown in Fig. 1.
        % In this research, we propose a method to present the sensation of flying with flapping wings bymanipulating the wings that grow from the back without using limbs movements.
        By not using limbs movements, it is possible to use hands and feet during the VR flight experience, such as throwing an object while flying, whick is expected to expand the range of the VR flight experience.


\section{EXPANSION OF BODY IMAGE}
        \fig{WingMan.pdf}{width=.9\hsize}{Flying with flapping virtual wings independent of the limbs}
        
        In this research, two elements are important: to make humans feel "wings" that do not originally exist, and to present the sensation of "flying with flapping with one's wings".  
        In order to present these sensations, we focus on the expantion of the body image.  

        \subsection{Body image}
                Humans have the ability to perceive their own body shape, which is called body image\cite{head1911sensory}.  
                It allows us to ditinguish between ourselves and others.

                Besides, there are two concepts that are closely related to the body image: sence of self-ownership and self-agency.
                Sence of self-ownership is the sensation or experience that one's own body parts belongs to one's own body.  
                Sence of self-agency is the sensation or experience that one is performing and action by oneself and that one is in control of the body parts.  
                
                The sense of self-ownership and self-agency are closely related to  the formation of the body image.  
                Therefore, it is considerd that the following elements in this research can be satisfied by flying with a virtual wings body image, that is by expanding the body image and operating.  

                \begin{itemize}
                        \item To make humans feel "wings" that do not originally exist (Sense of self-ownership)
                        \item To present the sensation of "flying with flapping with one's wings" (Sense of self-agency)
                \end{itemize}

        \subsection{Body image expantion}
                The body image may change dynamically to parts other than the self.  
                This is called body image expantion.  
                An example of body image expansion is to treat a tool (for example, a tennis rackaet or a baseball bat) held in the hand as if it were a part of one's own body without being aware of its shape, and hit the ball back\cite{botvinick1998rubber}.
                

                Body image expansion can be broadly categorized into two types: one is sensory remapping, such as the Rubber Hand Illusion (RHI), and the other is the dynamic expansion of the boy image during tool use mentioned above (Embodiment of tools).  

        \subsection{Rubber Hand Illusion}
                The Rubber Hand Illusion is the illusion that we feel the rubber hand as if it were our own hand.  
                It is an illusion phenomenon in which a person perceives a haptics stimulus on a rubber hand after gibing a synchronized haptics stimulus to a real hand hidden from field of vision and a rubber hand in front of the eyes for about 2 to 20 minutes.  
                One of the characteristics of RHI-based body image expansion is that the original body part and the remapped part cannnot coexist.

        \subsection{Embodiment of tools}
                There is a neurophysiological study on the embodiment of tools using Japanese macaque monkeys that showed the expansion of body image by tool use.  
                By observing the activity of bimodal nuerons with hand somatosensory receptors and visual receptors near the hand in the parietal cortex of Japanese macaque monkeys during tool use, and showed that the monkey's body image extendetd to the tip of the tool\cite{iriki1996coding}.
                

        \subsection{Body image expantion approach}
                In this research, we focus on the body image expantion to tools (embodiment of tools).

                It is known that tele-robots and avatars, which have similar degrees of freedom and dynamics to humans, can be recognized as part of the body by perfectly synchronizing their body movements, such as the generation of the sensation of being transported, as in the RHI, or the embodiment of tools.
                % ここ完全にDeepLにぶち込んだだけ,意味わからんから何も確認・修正出来ていない.
                
                It has also beeen shown that the temporal coincidence of sensory information (such as visual and haptics) is highly important in the generation of the RHI\cite{ehrsson2007experimental}.  
                
                Therefore, the temporal coincidence of sensory information is considerd to be important also in the embodiment of tools.  
                On the other hand, spatial coincidence is considered to be flexible, and there are cases where the subject responds as if the subject had struck his or her hand when hitting the rubber hand with RHI occurring\cite{armel2003projecting}.  

                \fig{maezuri-Method_of_body_image_expansion.pdf}{width=.8\hsize}{Method of body image expansion}

                As described above, body image expantion (embodiment of tools) can be expected by integratin multiple senses and matching the sensory information presented in time.  
                In this study , we try to expand the body image as shown in Fig. 2.  
                We attempted to integrate multiple senses by giving a instruction to move the wing from the human to the wing, and transmitting the sense of having wings, the sense of flying with wings, and the sense of air resistance.  
                From the above, we present the sensation of flying with flapping virtual wings independent of the limbs.

\section{HOW TO PRESENT WINGS INDEPENDENT OF LIMBS}
        
        \subsection{Information transmission from humans to virtual wings}
                First, we describe a method of giving instruction from a human to virtual wings to move the wings.
                There are several methods of presenting information from a human to a system in a VR space, such as using a controller, gestures, or biological signals.
                Since the purpose of this study is to control the wings with something other than the limbs, we use biological signals that can be measured without limbs movement, instead of controllers that mainly use the hands or gesture that require limbs movement.
                In addition, the virtual  wings is controlled by EMGs, which are easy to obtain numarical values among biological signals.

        \subsection{Information transmission from virtual wings to humans}
                Next, we describe a method of providing information from the virtual wings to a human. 
                The five senses can ve mentioned as senses that work on humans.
                Among them, visual and haptics senses are often used as the information that works on humans in body image expantion.  
                This indicates that the integration of visual and haptics senses is effective in body image expansion. 
                In this research, we present information from virtual wings using both visual and haptics senses.  

                \subsubsection{Presentation of visual information}
                
                        \fig{maezuri-Visual_presentation.pdf}{width=.8\hsize}{Presentation of visual information by using the virtual environment}

                        The visual presentation is performed by outputting images created by PC software (Unity) to a visual display as shown in Fig. 3. 
                        The output image shows it is moving in the air and wings growning from one's back.

                        We use an illusion called vection (self-induced self-motion sensation) to present the appearance of movement in the air\cite{bhalla1999visual}.
                        Vection is the illusion that a body moves in a direction opposite to the direction of motion of the stimulus when a uniform motion stimulus is presented over most of the visual field.  
                        
                        As for the presenting growing wings from the back, when the user looks back, one can see the part of the wings growing from the back.
                        Furthermore, when the wings are flapped, the appearance of expanding and closing the wings in the field of view is presented.

                \subsubsection{Presentation of haptics information}
                        In haptics presentation, by presenting the haptics sensation at the root of the wings, the sense of having wings, the sense of flying with wings, the sense of air resistance are presented.  
                        This encourages the expansion of the body image through the presentation of haptis sensations to a certain area, similar to the expansion of the body image to the tip of a stick held by the hand in the embodiment of tools.

                        Besides, we also prepare two haptics displays: one using vibration and the other using electrical stimulation.

                        Haptics display using vibration presents vibration by an eccentric motor.  
                        Haptics display using elecal stimulation is called EMS (Electro Myo Stimulation) is a method of stimulating muscle contraction by applying electrical stimulation to muscles and motor nerves to simulate haptics sensation.  

                        In this research, haptics sensation is generated by muscle contraction using EMS equipment, and information from the wings is presented.  


\section{PRE-EXPERIMENT USING THE PROPOSED METHOD}
% \section{SUBJECTIVE EVALUATION EXPERIMENT OF BODY IMAGE EXPANTION USING THE PROPOSED METHOD}
        Using the proposed method of body image expansion, we examine the methods of manipulation and presentation, the position of manipulation and presentation, and evaluate each combination.  

        \subsection{Exprimental environment for subjective evaluation}
                \fig{maezuri-Experiment_equipment_system.pdf}{width=.8\hsize}{Experimental environment system}
                We describe the experimental environment for the subjective evaluation experiment.
                As shown in Fig. 4, we use a system that sends the values measured by the electromyography device to the software (Unity) on the terminal, and operates the visual and haptics presentation devices from Unity.  

                \fig{maezuri-EMG_dev.pdf}{width=.8\hsize}{EMG devices (Left: Myo, Right: MyoWare)}
                First, two EMG measuring devices, Myo (Thalmic Labs, Fig. 5. left) and MyoWare (Advancer Technologies, Fig. 5. right) , are prepared to compare manipulation by gesture and force.  

                \fig{EMG_device_HV-F122.pdf}{width=0.7\hsize}{EMG device (Omron HV-F122)}
                Next, for haptics display, we prepare haptics presentation devices using vibration and EMS.  
                The vibration function of Myo is used for vibration haptics presentation, and the Omron HV-F122 (Fig. 6) , a low-frequency treatment device, is used for EMS haptivs display.  

                % \fig{exprtiment-VirtualWings_skeleton.pdf}{width=0.7\hsize}{Virtual Wings skeleton}
                Then, for the visual presentation device, two types of visual presentation will be prepared, one using an LCD monitor and the other using a Head Mounted Display (HMD) , in order to compare the third-person perspective and the first-person perspective.  
                The LCD monitor is GW2765HT (BenQ), and the HMD is Quest from Meta. 
                % The virtual wings used for visual presentation is shown in Fig. 7.  

        \subsection{Examination of operation and presentation methods}
                \begin{table}[tb]
                        \tablabel{comparison}
                        \begin{center}
                        \caption{Comparison items}
                        \scalebox{0.70}{
                        \begin{tabular}{l|c|c|c}
                                \hline
                                Comparison item & Wings operation & Haptics display & Visual display\\
                                \hline
                                Wings operation & Gesture/Strengthen & Vibration & TPP \\
                                Haptics display & Strengthen & Vibration/EMS & TPP \\
                                Visual display & Strengthen & EMS & TPP/FPP \\
                                \hline
                        \end{tabular}
                        }          
                        \end{center}
                \end{table}

                Table 1 shows the comparison items, where FPP is First Person Perspective and TPP is Third Person Perspective.  

                \subsubsection{Comparison of wings operation methods}
                        First, we compare the manipulation methods of the wings.  
                        Manipulation methods using EMG can be classified in to gesture (dynamic muscle contraction\cite{thistle1967isokinetic} ), which are movements accompained by joint movements, and manipulation by force or static muscle contraction, which does not involve joint movements.  
                        
                        Fixing the conditions of haptics and visual presentation and comparing manipulation by gesture and force (Table 1).

                        \fig{Manipulation_of_VirtualWings_using_Myo.pdf}{width=0.8\hsize}{Manipulation of virtual wings skeleton using Myo}
                        \fig{Manipulation_of_VirtualWings_using_MyoWare.pdf}{width=0.8\hsize}{Manipulation of virtual wings skeleton using MyoWare}
                
                        When Myo is used as the myoelectric measurement device (manipulated by gesture), the virtual wings is designed so that when the wrist is bent inward, the wings also flaps inward, and when the wrist is opened outward, the wings also opens outward, as shown in Fig. 7.  
                        The haptics sensation is presented in such a way that the Myo attached to the forearm vibrates in time with the inward flapping of the wings.  

                        In the case of using MyoWare (manipulation by force), the wings are designed to close when forces is applied and to open when relaxed, as shown in Fig. 8.    
                        The haptics presentation is made by vibrating Myo attached to the arm when the wings flap inward.  
                        MyoWare measures the forearm, chest, and shoulder as the measurement sites, where the forceful motion is easy.  

                        As a result of the verification, we confirmed that it is possible to operate the virtual wings using force as well as gestures, which is a common method of operating VR applications, as intended by the pilot.  


                \subsubsection{Comparison of haptics presentation methods}
                        Next, we compare the haptics presentation methods using vibration and EMS.  

                        We fixed the conditions of wings manipulation and visual presentation, and compared the presentation using vibration as a haptics sense with that using electrical stimulation (Table 1).  
                        We used MyoWare to manipulate the wings so that they closed when the forearms, chest, and shoulders were relaxed.  
                        For the visual presentation, the wings movements (Fig. 9) are shown on the LCD display from a third-person perspective.  

                        Table 2 and Table 3 show the subjective evaluation of the information presented from the virtual wings to the human during the experiment using the Likert scale in five levels when vibration and electrical stimulation were used as haptics presentation.  

                        \begin{table}[tb]
                                \tablabel{exp1_result}
                                \begin{center}
                                    \caption{Results of an experiment using vibration as a haptics presentation}
                                    \scalebox{0.70}{
                                      \begin{tabular}{l|c|c|c}
                                          \hline
                                          Position(Sensing/Haptics) & Arm/Arm & Chest/Arm & Shoulder/Arm\\
                                          \hline
                                          Sense of having wings & 1 & 1 & 2 \\
                                          Sense of meneuvering the wings & 4 & 4 & 4 \\
                                          Sense of air resistance & 4 & 4 & 4 \\
                                          Sense of flying with wings & 1 & 2 & 2 \\
                                          \hline
                                      \end{tabular}
                                    }
                                \end{center}
                        \end{table}
                            
                            \begin{table}[t]
                                \tablabel{exp2_result}
                                \begin{center}
                                    \caption{Results of experiments using EMS as haptics presentation}
                                    \scalebox{0.70}{
                                      \begin{tabular}{l|c|c|c}
                                          \hline
                                          % 触覚提示位置(腕)
                                          Position(Sensing/Haptics) & Arm/Arm & Chest/Arm & Shoulder/Arm \\
                                          \hline
                                          Sense of having wings & 1 & 2 & 2 \\
                                          Sense of meneuvering the wings & 3 & 3 & 4\\
                                          Sense of air resistance & 3 & 3 & 3 \\
                                          Sense of flying with wings & 2 & 2 & 3 \\
                                          \hline\hline
                          
                                          % 筋電取得位置(胸)
                                          Position(Sensing/Haptics) & Arm/Abs & Chest/Abs & Shoulder/Abs \\
                                          \hline
                                          Sense of having wings & 3 & 3 & 3 \\
                                          Sense of meneuvering the wings & 3 & 3 & 4\\
                                          Sense of air resistance & 4 & 4 & 4 \\
                                          Sense of flying with wings & 3 & 4 & 3 \\                        
                                          \hline\hline
                          
                                          % 筋電取得位置(腹)
                                          Position(Sensing/Haptics) & Arm/Back & Chest/Back & Shoulder/Back  \\
                                          \hline                        
                                          Sense of having wings & 5 & 5 & 5 \\                        
                                          Sense of meneuvering the wings & 3 & 4 & 5 \\
                                          Sense of air resistance & 4 & 4 & 4\\
                                          Sense of flying with wings & 4 & 5 & 5 \\
                                          \hline\hline
                                      \end{tabular}
                                    }
                                \end{center}
                        \end{table}
                        
                        The first row of Table 2 and Table 3 shown the results of the haptics presentaion on the forearm, respectively.  
                        Teh first row of Table 2 and Table 3 show the results of haptics presentation to the forarm.  
                        Therefore, it can be said that not only vibration but also electric stimulation is useful for haptics presentation.  
                        
                        In addition, Table 3 shows that the overall evaluation of both myoelectric measurement and haptics presentation was higher in the torso than in the limbs (arms).  
                        Therefore, it is thought that the sensation of flying with flapping can be presented more strongly when both myoelectric measurement and haptics presentation are performed on the torso than on the limbs.  


                \subsubsection{Comparison of visual presentation methods}
                        Finally, we compare the visual presentation methods.  
                        We prepared an LCD display and an HMD as visual presentation devices, and presented images from the third-person and first-person viewpoints, respectively.  

                        The operation method of the wings and the conditions for haptics presentation are fixed, and a comparison is made when the third-person viewpoint and the first-person viewpoint are used as visual presentation (Table 1).  
                        The wings are designed to close when the forearms, chest, and shoulder are Strengthen, and to open when they are relaxed.  
                        Haptics sensation is presented to the forearms, chest, abdomen, and back using EMS equipment (Fig. 6).  

                        We found that the first-person perspective induced a stronger sense of wings growing from one's own body than the third-person view, and that the third-person view induced a sense of manipulating remote wings rathrer than one's own wings, and a sense of remapping like the RHI rathrer than embodiment of tools, it is not a sense of embodiment. 
                        The first-person perspective was found to be more effective than the third-person perspective in embodiment of tools.  

        
        \subsection{Examination of operation and presentation position}
                As a result of examining the operation and presentation methods, we found that the presentation of the sensation of flapping wings differs depending on the operation position (myoelectric measurement position) and the haptics presentation position.  
                Therefore, in order to efficiently compare the differences between the two positions, we selected candidates for the myoelectric measurement and haptics presentation positions.  
                        
                \subsubsection{Examination of operation posotion}7
                        In the case of the body, the chest, abdomen, shoulders, and buttocks are the most common sites for static muscle contraction. 
                        In myoelectric measurement, if there is a lot of subcutaneous fat at the measurement point, the amplitude of the myoelectric potential is attenuated and becomes unclear. 
                        Therefore, it is necessary to select an area with relatively little subcutaneous fat as the myoelectric measurement location. 
                        In order to compare the operation of the virtual wing by gesture and force, three types of dynamic muscle contraction of the biceps brachii muscle are used as the myoelectric measurement positions: chest and shoulder.
                        % \cite{白石恵1992筋電位多点計測による体幹背部の神経支配帯の分布}

                \subsubsection{Examination of presentation posotion}
                        From Table 2, it was found that the evaluation was higher when the haptics presentation was made on the torso than on the extremities. 
                        Therefore, we investigate which part of the torso is the most effective for haptics presentation. 
                        It is known that the hand occupies a large proportion of the human sensory cortex, while the torso occupies a small proportion of the sensory cortex\cite{penfield1950cerebral}. 
                        Accordingly, we thought that it would be better to classify the torso parts into large groups instead of small ones, so that we can investigate the differences in sensation between the parts. 
                        Consequently, we divided the torso into the upper and lower parts, and the front and back parts (chest, abdomen, back, and waist).        


\section{SUBJECT EXPERIMENT TO VERIFY THE DIFFERENCE BY POSITION}
% \section{A SUBJECT EXPERIMENT TO EVALUATE THE DIFFERENCE OF BODY IMAGE EXPANTION BY POSITION BASED ON A SUBJECTIVE EVALUATION EXPERIMENT}
%%%%%% タイトル長すぎ問題 %%%%%%%%%

        \subsection{Purpose of the experiment}
                In the previous section, we confirmed that there is a difference in the sensation of flapping wings depending on the myoelectric measurement position and the haptics presentation position.  
                In this section, we verify the difference in the sensation (degree of body image expansion) between the myoelectric measurement position and the haptics presentation position from the subject experiment.  

        \subsection{Expreimantal environment for conducting subject experiments}
                \fig{VirtualWingsV2.png}{width=0.8\hsize}{Virtual Wings}
                This section describes the experimental environment in which the subject experiment is conducted.  
                In the subject experiment, as in the subjective evaluation experiment in Section 4, the experiment is conducted in a system with the environment as shown in Figure 4.  

                We use MyoWare (Fig. 5) as the myoelectric measurement devices.  
                The acquired EMG values are sent to Unity to control the visual and haptics presentation devices.  

                The HMD (VIVE PRO EYE (HTC)) is used as the visual presentation device, and the virtual wings (Fig. 9) is presented from a first-person perspective.  

                
                \fig{maezuri-Haptics_dev.pdf}{width=.8\hsize}{Haptics display device (Left: Vibration, Right: EMS)}
                Fig. 10 shows the haptics presentation devices.  
                Fig. 10 left shows a device that controls an eccentric motor with Arduino to give vibration.  
                Fig. 10 right shows a device that controls a low-frequency therapy device (Omron HV-F127) with Arduino and provides electrical stimulation.  

                % The physical environment of the experiment is described below.  
                In the VR space of this experiment, the gravitational acceleration $g^{\prime}$ is set to be equivalent to that of the moon 
                % (Eq. (1)) 
                to facilitate flight ($g^{\prime}=1.62$).
                % \begin{eqnarray}
                %         \equlabel{gravity}
                %         g^{\prime}=1.62\;[\mathrm{m/s^{2}}]
                % \end{eqnarray}
                It also gives an aerodynamic drag 
                % $\bm{R}\;[\mathrm{N}]$ 
                proportional to the speed
                % , as in Eq. (2), 
                during flight.  
                % \begin{eqnarray}
                %         \equlabel{air_resistance}
                %         \bm{R}=k\bm{v} \;[\mathrm{N}]
                % \end{eqnarray}
                % $k$ is a constant of proportionality ($k=5.0$).  

                
                \fig{Movement_of_VirtualWingsV2.pdf}{width=.8\hsize}{Movement of Virtual Wings}
                A specific flight method in VR space is described.  
                First, the wings operation method is designed so that the virtual wings flaps inward while the operation position (electromyographic measurement position) is being applied, and the virtual wings expands when it is relaxed (Fig. 11).  
                By flapping the virtual wings, a force 
                % $\bm{F}\;[\mathrm{N}]$ 
                is generated in the direction of travel and ascent, allowning the aircraft to fly.  
                % The force generated at this time, $\bm{F}\;[\mathrm{N}]$, obeys Eq. (3).  
                % \begin{eqnarray}
                %         \equlabel{force}
                %         \bm{F}=\frac{a}{\bm{l}}  \tanh\left(\,\frac{x}{a}\,\right) \;[\mathrm{N}]
                % \end{eqnarray}
                % \begin{eqnarray}
                %         \equlabel{force_prop}
                %         \bm{l}=\begin{pmatrix}400 \\ 500 \end{pmatrix}
                % \end{eqnarray}
                % $l\;$ is a constant of propotionality (1st line: direction of travel, 2nd line: direction of ascent). 
                % $a\;$[N] is the maximum measurable EMG potential ($a=1024$), $x\;$[N] is the measured EMG potential.  

                When the virtual wings are flapped inward, a haptics presentation is made.  
                The intensity of the haptics presentation at this time 
                % $P\;$[N] should change Eq. (5) 
                according to the measured muscle potential.  
                % \begin{eqnarray}
                %         \equlabel{haptics}
                %         P=a \tanh\left(\,\frac{x}{a}\,\right) \;[\mathrm{N}]
                % \end{eqnarray}

        
        \subsection{Experimental method}
                Three types of EMG measurement positions (biceps brachii (dynamic contraction) / pectoralis major (static contraction) / trapezius (static contraction)), four types of haptics presentation positions (chest / abdomen / waist / back), and two types of haptics presentation devices (vibration / electricity) are compared.  
                % The EMG measurement positions are biceps brachii (dynamic contraction), pectoralis major (static contraction), and trapezius (static contraction), the haptics presentation positions are chest, abdomen, waist, and back, and the haptics presentation types are vibration and electricity.
                
                The procedure of the experiment is shown below.  
                \begin{itemize}
                        \item The EMG measurement position is fixed, and the haptics presentation position is changed for comparison.
                        \item The haptics sensation presentation position is fixed, and the myoelectric measurement position is changed for comparison.
                        \item Change the haptics presentation device to EMS and repeat above process.
                \end{itemize}

                The questionnaire was filled out using a 9-point Likert scale. The content of the questionnaire is shown below. 
                \begin{itemize}
                        \item Comparison of haptics presentation position. 
                        \item Comparison of EMG measurement position.
                                %\\ (Ask the above two questions twice, once using vibration and once using EMS as the haptics presentation.)
                        \item Which is better, vibration or electrical stimulation?
                        \item Which is more important, the EMG measurement position or the haptics presentation position?
                        \item How well did you feel the sensation of flying with flapping with wings growing from your back?
                \end{itemize}
                
        \subsection{Experimental results and discussion}
                \figpos{subjectexp-result_haptics_pos.png}{width=0.8\hsize}{Comparison of haptics display positions}{t}

                \figpos{subjectexp-result_emg_pos.png}{width=.8\hsize}{Comparison of EMG measuring position}{t}

                % \figpos{subjectexp-result_vib_or_ems.png}{width=.8\hsize}{Comparison of Vibration and EMS}{t}

                \figpos{subjectexp-result_mesure_or_presen.png}{width=.8\hsize}{Comparison of measuring position and haptics presentation position}{t}

                Fig. 12 shows the mean values of the evaluation when the myoelectric measurement position is fixed and the haptics presentation position is compared.
                From the figure, it can be seen that the evaluation is high in the order of back, waist, chest, and abdomen regardless of whether the presenting device is vibration or electric stimulation.
                This indicates that the front and back of the torso have more influence on the evaluation of the body image expansion than the absolute distance from the virtual wing position given by the visual presentation.

                Fig. 13 shows the average of the evaluations when the haptics presentation position is fixed and the myoelectric measurement position is compared.
                From the figure, it can be confirmed that the evaluation of the virtual wing manipulation by the shoulder is the highest regardless of whether the presenting device is vibration or electric stimulation. 
                There was no significant difference between the evaluations of arms (dynamic muscle contraction) and chest (static muscle contraction), which are close to the position of the muscles. 
                This suggests that there is no significant difference in the sensations elicited by gestures (dynamic muscle contraction) and force (static muscle contraction) in body image expansion. 
                In addition, the difference between the two gestures is small, even though the arm gesture is an intuitive action and the chest strain is an unusual and difficult action to perform. 
                Therefore, it can be expected that the training of the manipulation by force will be more effective than the manipulation by gesture in enhancing the body image of the virtual wings.

                % Fig. 14 shows a comparison of the haptics presentation of vibration and electrical stimulation. 
                Seven of the subjects answered that vibration was better for haptics presentation, and the remaining eight answered that electric stimulation was better. 
                In addition to vibration, electrical stimulation is also useful for haptics presentation. 
                Fig. 12 and Fig. 13 show that the evaluation of the presentation using the electric stimulus was higher in general. 
                This can be attributed to the fact that the haptics sensation presentation using vibration stimulates the skin surface (superficial sensation), while the haptics sensation presentation using electrical stimulation easily transmits the stimulation to the muscles (deep sensation) in addition to the skin surface, thus increasing the mode of sensation presented.

                Fig. 14 shows the results of the question whether the myoelectric measurement position or the haptics presentation position was more important in presenting the sensation of flapping wings independent of the limb. 
                As shown in the figure, most of the subjects answered that the myoelectric measurement position was important in both cases of vibration and electric stimulation.
                The myoelectric measurement position is the part that acquires the information that works from the human while the haptics presentation position is the response from the wing.
                Therefore, it can be understood that the subjects place importance on the sense of moving the wings by themselves, and the sense of self-subjectivity.
                In the example of the RHI study, we mentioned that the spatial correspondence of the haptics presentation position is flexible in body image expansion. However, from the experimental results, it is considered that the spatial agreement of the myoelectric measurement position (manipulation position by force) is important in body image expansion.

                Finally, when asked if they could feel the sensation of flying with the wings sprouting from their backs, they were able to obtain an average rating of 7.11. 
                Therefore, it was found that the method proposed in this study can be used to extend the body image of virtual wings.


\section{CONCLUSION}
        In this paper, we focused on the study of giving the sensation of flying by moving the wings. 
        We proposed a method of presenting the sensation of manipulating the wings without using the limbs and a method of transmitting the force acting on the wings to humans in VR space. 
        We proposed a method of presenting the sensation of manipulating the wings without using the limbs, and a method of transmitting the force acting on the wings in the VR space to humans. 
        From the subject experiments, it is considered that the factors of the front and rear surfaces of the body are more important than the absolute distance in presenting the haptics sensation. 
        In addition, we proposed a method to evaluate the position of the myoelectric measurement by force. 
        In addition, we showed that the evaluation was generally higher for the presentation using electric stimuli. 
        Finally, we confirmed that the proposed method can be used to extend the body image of virtual wings.
    
        In the future, we would like to increase the amount of information presented from the virtual wing to the human, and improve the equipment used in the experiments, such as changing the method of myoelectric measurement from single point measurement to multi-point measurement to obtain stable values. 
        In this paper, the myoelectric measurement and haptics presentation positions were narrowed down to some extent for comparison, but by increasing the number of candidates, it will be possible to learn more about the differences by position.


%%%%%%%%% template %%%%%%%%%%%%
% \section{HOGE-FUGA}
% This template provides authors with most of the formatting specifications needed for preparing electronic versions of their papers. All standard paper components have been specified for three reasons: (1) ease of use when formatting individual papers, (2) automatic compliance to electronic requirements that facilitate the concurrent or later production of electronic products, and (3) conformity of style throughout a conference proceedings. Margins, column widths, line spacing, and type styles are built-in; examples of the type styles are provided throughout this document and are identified in italic type, within parentheses, following the example. Some components, such as multi-leveled equations, graphics, and tables are not prescribed, although the various table text styles are provided. The formatter will need to create these components, incorporating the applicable criteria that follow.

% \section{PROCEDURE FOR PAPER SUBMISSION}

% \subsection{Selecting a Template (Heading 2)}

% First, confirm that you have the correct template for your paper size. This template has been tailored for output on the US-letter paper size. 
% It may be used- for A4 paper size if the paper size setting is suitably modified.

% \subsection{Maintaining the Integrity of the Specifications}

% The template is used to format your paper and style the text. All margins, column widths, line spaces, and text fonts are prescribed; please do not alter them. You may note peculiarities. For example, the head margin in this template measures proportionately more than is customary. This measurement and others are deliberate, using specifications that anticipate your paper as one part of the entire proceedings, and not as an independent document. Please do not revise any of the current designations

% \section{MATH}

% Before you begin to format your paper, first write and save the content as a separate text file. Keep your text and graphic files separate until after the text has been formatted and styled. Do not use hard tabs, and limit use of hard returns to only one return at the end of a paragraph. Do not add any kind of pagination anywhere in the paper. Do not number text heads-the template will do that for you.

% Finally, complete content and organizational editing before formatting. Please take note of the following items when proofreading spelling and grammar:

% \subsection{Abbreviations and Acronyms} Define abbreviations and acronyms the first time they are used in the text, even after they have been defined in the abstract. Abbreviations such as IEEE, SI, MKS, CGS, sc, dc, and rms do not have to be defined. Do not use abbreviations in the title or heads unless they are unavoidable.

% \subsection{Units}

% \begin{itemize}

% \item Use either SI (MKS) or CGS as primary units. (SI units are encouraged.) English units may be used as secondary units (in parentheses). An exception would be the use of English units as identifiers in trade, such as ?3.5-inch disk drive?.
% \item Avoid combining SI and CGS units, such as current in amperes and magnetic field in oersteds. This often leads to confusion because equations do not balance dimensionally. If you must use mixed units, clearly state the units for each quantity that you use in an equation.
% \item Do not mix complete spellings and abbreviations of units: ?Wb/m2? or ?webers per square meter?, not ?webers/m2?.  Spell out units when they appear in text: ?. . . a few henries?, not ?. . . a few H?.
% \item Use a zero before decimal points: ?0.25?, not ?.25?. Use ?cm3?, not ?cc?. (bullet list)

% \end{itemize}


% \subsection{Equations}

% The equations are an exception to the prescribed specifications of this template. You will need to determine whether or not your equation should be typed using either the Times New Roman or the Symbol font (please no other font). To create multileveled equations, it may be necessary to treat the equation as a graphic and insert it into the text after your paper is styled. Number equations consecutively. Equation numbers, within parentheses, are to position flush right, as in (1), using a right tab stop. To make your equations more compact, you may use the solidus ( / ), the exp function, or appropriate exponents. Italicize Roman symbols for quantities and variables, but not Greek symbols. Use a long dash rather than a hyphen for a minus sign. Punctuate equations with commas or periods when they are part of a sentence, as in

% $$
% \alpha + \beta = \chi \eqno{(1)}
% $$

% Note that the equation is centered using a center tab stop. Be sure that the symbols in your equation have been defined before or immediately following the equation. Use ?(1)?, not ?Eq. (1)? or ?equation (1)?, except at the beginning of a sentence: ?Equation (1) is . . .?

% \subsection{Some Common Mistakes}
% \begin{itemize}


% \item The word ?data? is plural, not singular.
% \item The subscript for the permeability of vacuum ?0, and other common scientific constants, is zero with subscript formatting, not a lowercase letter ?o?.
% \item In American English, commas, semi-/colons, periods, question and exclamation marks are located within quotation marks only when a complete thought or name is cited, such as a title or full quotation. When quotation marks are used, instead of a bold or italic typeface, to highlight a word or phrase, punctuation should appear outside of the quotation marks. A parenthetical phrase or statement at the end of a sentence is punctuated outside of the closing parenthesis (like this). (A parenthetical sentence is punctuated within the parentheses.)
% \item A graph within a graph is an ?inset?, not an ?insert?. The word alternatively is preferred to the word ?alternately? (unless you really mean something that alternates).
% \item Do not use the word ?essentially? to mean ?approximately? or ?effectively?.
% \item In your paper title, if the words ?that uses? can accurately replace the word ?using?, capitalize the ?u?; if not, keep using lower-cased.
% \item Be aware of the different meanings of the homophones ?affect? and ?effect?, ?complement? and ?compliment?, ?discreet? and ?discrete?, ?principal? and ?principle?.
% \item Do not confuse ?imply? and ?infer?.
% \item The prefix ?non? is not a word; it should be joined to the word it modifies, usually without a hyphen.
% \item There is no period after the ?et? in the Latin abbreviation ?et al.?.
% \item The abbreviation ?i.e.? means ?that is?, and the abbreviation ?e.g.? means ?for example?.

% \end{itemize}


% \section{USING THE TEMPLATE}

% Use this sample document as your LaTeX source file to create your document. Save this file as {\bf root.tex}. You have to make sure to use the cls file that came with this distribution. If you use a different style file, you cannot expect to get required margins. Note also that when you are creating your out PDF file, the source file is only part of the equation. {\it Your \TeX\ $\rightarrow$ PDF filter determines the output file size. Even if you make all the specifications to output a letter file in the source - if your filter is set to produce A4, you will only get A4 output. }

% It is impossible to account for all possible situation, one would encounter using \TeX. If you are using multiple \TeX\ files you must make sure that the ``MAIN`` source file is called root.tex - this is particularly important if your conference is using PaperPlaza's built in \TeX\ to PDF conversion tool.

% \subsection{Headings, etc}

% Text heads organize the topics on a relational, hierarchical basis. For example, the paper title is the primary text head because all subsequent material relates and elaborates on this one topic. If there are two or more sub-topics, the next level head (uppercase Roman numerals) should be used and, conversely, if there are not at least two sub-topics, then no subheads should be introduced. Styles named ?Heading 1?, ?Heading 2?, ?Heading 3?, and ?Heading 4? are prescribed.

% \subsection{Figures and Tables}

% Positioning Figures and Tables: Place figures and tables at the top and bottom of columns. Avoid placing them in the middle of columns. Large figures and tables may span across both columns. Figure captions should be below the figures; table heads should appear above the tables. Insert figures and tables after they are cited in the text. Use the abbreviation ?Fig. 1?, even at the beginning of a sentence.

% \begin{table}[h]
% \caption{An Example of a Table}
% \label{table_example}
% \begin{center}
% \begin{tabular}{|c||c|}
% \hline
% One & Two\\
% \hline
% Three & Four\\
% \hline
% \end{tabular}
% \end{center}
% \end{table}


%    \begin{figure}[thpb]
%       \centering
%       \framebox{\parbox{3in}{We suggest that you use a text box to insert a graphic (which is ideally a 300 dpi TIFF or EPS file, with all fonts embedded) because, in an document, this method is somewhat more stable than directly inserting a picture.
% }}
%       %\includegraphics[scale=1.0]{figurefile}
%       \caption{Inductance of oscillation winding on amorphous
%        magnetic core versus DC bias magnetic field}
%       \label{figurelabel}
%    \end{figure}
   

% Figure Labels: Use 8 point Times New Roman for Figure labels. Use words rather than symbols or abbreviations when writing Figure axis labels to avoid confusing the reader. As an example, write the quantity ?Magnetization?, or ?Magnetization, M?, not just ?M?. If including units in the label, present them within parentheses. Do not label axes only with units. In the example, write ?Magnetization (A/m)? or ?Magnetization {A[m(1)]}?, not just ?A/m?. Do not label axes with a ratio of quantities and units. For example, write ?Temperature (K)?, not ?Temperature/K.?

% \section{CONCLUSIONS}

% A conclusion section is not required. Although a conclusion may review the main points of the paper, do not replicate the abstract as the conclusion. A conclusion might elaborate on the importance of the work or suggest applications and extensions. 

\addtolength{\textheight}{-12cm}   % This command serves to balance the column lengths
                                  % on the last page of the document manually. It shortens
                                  % the textheight of the last page by a suitable amount.
                                  % This command does not take effect until the next page
                                  % so it should come on the page before the last. Make
                                  % sure that you do not shorten the textheight too much.

%%%%%%%%%%%%%%%%%%%%%%%%%%%%%%%%%%%%%%%%%%%%%%%%%%%%%%%%%%%%%%%%%%%%%%%%%%%%%%%%



%%%%%%%%%%%%%%%%%%%%%%%%%%%%%%%%%%%%%%%%%%%%%%%%%%%%%%%%%%%%%%%%%%%%%%%%%%%%%%%%



%%%%%%%%%%%%%%%%%%%%%%%%%%%%%%%%%%%%%%%%%%%%%%%%%%%%%%%%%%%%%%%%%%%%%%%%%%%%%%%%
% \section*{APPENDIX}

% Appendixes should appear before the acknowledgment.

% \section*{ACKNOWLEDGMENT}

% The preferred spelling of the word ?acknowledgment? in America is without an ?e? after the ?g?. Avoid the stilted expression, ?One of us (R. B. G.) thanks . . .?  Instead, try ?R. B. G. thanks?. Put sponsor acknowledgments in the unnumbered footnote on the first page.

% \citation{IEEEexample:shellCTANpage}

%%%%%%%%%%%%%%%%%%%%%%%%%%%%%%%%%%%%%%%%%%%%%%%%%%%%%%%%%%%%%%%%%%%%%%%%%%%%%%%%

% References are important to the reader; therefore, each citation must be complete and correct. If at all possible, references should be commonly available publications.


% \nocite{*}
% \begin{thebibliography}{99}

% \bibitem{c1} G. O. Young, ?Synthetic structure of industrial plastics (Book style with paper title and editor),? 	in Plastics, 2nd ed. vol. 3, J. Peters, Ed.  New York: McGraw-Hill, 1964, pp. 15?64.
% \bibitem{c2} W.-K. Chen, Linear Networks and Systems (Book style).	Belmont, CA: Wadsworth, 1993, pp. 123?135.
% \bibitem{c3} H. Poor, An Introduction to Signal Detection and Estimation.   New York: Springer-Verlag, 1985, ch. 4.
% \bibitem{c4} B. Smith, ?An approach to graphs of linear forms (Unpublished work style),? unpublished.
% \bibitem{c5} E. H. Miller, ?A note on reflector arrays (Periodical style?Accepted for publication),? IEEE Trans. Antennas Propagat., to be publised.
% \bibitem{c6} J. Wang, ?Fundamentals of erbium-doped fiber amplifiers arrays (Periodical style?Submitted for publication),? IEEE J. Quantum Electron., submitted for publication.
% \bibitem{c7} C. J. Kaufman, Rocky Mountain Research Lab., Boulder, CO, private communication, May 1995.
% \bibitem{c8} Y. Yorozu, M. Hirano, K. Oka, and Y. Tagawa, ?Electron spectroscopy studies on magneto-optical media and plastic substrate interfaces(Translation Journals style),? IEEE Transl. J. Magn.Jpn., vol. 2, Aug. 1987, pp. 740?741 [Dig. 9th Annu. Conf. Magnetics Japan, 1982, p. 301].
% \bibitem{c9} M. Young, The Techincal Writers Handbook.  Mill Valley, CA: University Science, 1989.
% \bibitem{c10} J. U. Duncombe, ?Infrared navigation?Part I: An assessment of feasibility (Periodical style),? IEEE Trans. Electron Devices, vol. ED-11, pp. 34?39, Jan. 1959.
% \bibitem{c11} S. Chen, B. Mulgrew, and P. M. Grant, ?A clustering technique for digital communications channel equalization using radial basis function networks,? IEEE Trans. Neural Networks, vol. 4, pp. 570?578, July 1993.
% \bibitem{c12} R. W. Lucky, ?Automatic equalization for digital communication,? Bell Syst. Tech. J., vol. 44, no. 4, pp. 547?588, Apr. 1965.
% \bibitem{c13} S. P. Bingulac, ?On the compatibility of adaptive controllers (Published Conference Proceedings style),? in Proc. 4th Annu. Allerton Conf. Circuits and Systems Theory, New York, 1994, pp. 8?16.
% \bibitem{c14} G. R. Faulhaber, ?Design of service systems with priority reservation,? in Conf. Rec. 1995 IEEE Int. Conf. Communications, pp. 3?8.
% \bibitem{c15} W. D. Doyle, ?Magnetization reversal in films with biaxial anisotropy,? in 1987 Proc. INTERMAG Conf., pp. 2.2-1?2.2-6.
% \bibitem{c16} G. W. Juette and L. E. Zeffanella, ?Radio noise currents n short sections on bundle conductors (Presented Conference Paper style),? presented at the IEEE Summer power Meeting, Dallas, TX, June 22?27, 1990, Paper 90 SM 690-0 PWRS.
% \bibitem{c17} J. G. Kreifeldt, ?An analysis of surface-detected EMG as an amplitude-modulated noise,? presented at the 1989 Int. Conf. Medicine and Biological Engineering, Chicago, IL.
% \bibitem{c18} J. Williams, ?Narrow-band analyzer (Thesis or Dissertation style),? Ph.D. dissertation, Dept. Elect. Eng., Harvard Univ., Cambridge, MA, 1993. 
% \bibitem{c19} N. Kawasaki, ?Parametric study of thermal and chemical nonequilibrium nozzle flow,? M.S. thesis, Dept. Electron. Eng., Osaka Univ., Osaka, Japan, 1993.
% \bibitem{c20} J. P. Wilkinson, ?Nonlinear resonant circuit devices (Patent style),? U.S. Patent 3 624 12, July 16, 1990. 

% \end{thebibliography}



\bibliographystyle{plain}
\bibliography{reference.bib}
\ \
\end{document}
